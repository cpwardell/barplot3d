%\VignetteDepends{knitr}
%\VignetteIndexEntry{An introduction to gaucho}
%\VignetteCompiler{knitr}
%\VignetteEngine{knitr::knitr}

\documentclass{article}\usepackage[]{graphicx}\usepackage[]{color}
%% maxwidth is the original width if it is less than linewidth
%% otherwise use linewidth (to make sure the graphics do not exceed the margin)
\makeatletter
\def\maxwidth{ %
  \ifdim\Gin@nat@width>\linewidth
    \linewidth
  \else
    \Gin@nat@width
  \fi
}
\makeatother

\definecolor{fgcolor}{rgb}{0.345, 0.345, 0.345}
\newcommand{\hlnum}[1]{\textcolor[rgb]{0.686,0.059,0.569}{#1}}%
\newcommand{\hlstr}[1]{\textcolor[rgb]{0.192,0.494,0.8}{#1}}%
\newcommand{\hlcom}[1]{\textcolor[rgb]{0.678,0.584,0.686}{\textit{#1}}}%
\newcommand{\hlopt}[1]{\textcolor[rgb]{0,0,0}{#1}}%
\newcommand{\hlstd}[1]{\textcolor[rgb]{0.345,0.345,0.345}{#1}}%
\newcommand{\hlkwa}[1]{\textcolor[rgb]{0.161,0.373,0.58}{\textbf{#1}}}%
\newcommand{\hlkwb}[1]{\textcolor[rgb]{0.69,0.353,0.396}{#1}}%
\newcommand{\hlkwc}[1]{\textcolor[rgb]{0.333,0.667,0.333}{#1}}%
\newcommand{\hlkwd}[1]{\textcolor[rgb]{0.737,0.353,0.396}{\textbf{#1}}}%

\usepackage{framed}
\makeatletter
\newenvironment{kframe}{%
 \def\at@end@of@kframe{}%
 \ifinner\ifhmode%
  \def\at@end@of@kframe{\end{minipage}}%
  \begin{minipage}{\columnwidth}%
 \fi\fi%
 \def\FrameCommand##1{\hskip\@totalleftmargin \hskip-\fboxsep
 \colorbox{shadecolor}{##1}\hskip-\fboxsep
     % There is no \\@totalrightmargin, so:
     \hskip-\linewidth \hskip-\@totalleftmargin \hskip\columnwidth}%
 \MakeFramed {\advance\hsize-\width
   \@totalleftmargin\z@ \linewidth\hsize
   \@setminipage}}%
 {\par\unskip\endMakeFramed%
 \at@end@of@kframe}
\makeatother

\definecolor{shadecolor}{rgb}{.97, .97, .97}
\definecolor{messagecolor}{rgb}{0, 0, 0}
\definecolor{warningcolor}{rgb}{1, 0, 1}
\definecolor{errorcolor}{rgb}{1, 0, 0}
\newenvironment{knitrout}{}{} % an empty environment to be redefined in TeX

\usepackage{alltt}

% LaTeX packages 
\usepackage{hyperref} % For hyperlinks\
\usepackage{float} % for floating images
\usepackage{alltt} % for verbatim blocks of text
\usepackage{verbatim} % to allow multi-line block comments
\usepackage{color} % to define custom colours
\usepackage{a4wide} % gives us wider pages
\usepackage[backend=bibtex,sorting=none]{biblatex} % bibliography package

% Setting up bibliography
\bibliography{barplot3d_bib}

% This block sets all links to be black and removes the hideous coloured boxes around them
\hypersetup{
    colorlinks=true,
    linkcolor=black,
    citecolor=black,
    filecolor=black,
    urlcolor=black,
}
\IfFileExists{upquote.sty}{\usepackage{upquote}}{}
\begin{document}

\title{Barplot 3D Examples}
\author{
  Christopher P Wardell\\
  r@cpwardell.com
}
\maketitle

This vignette gives examples of how to use the barplot3d package.  It contains several worked examples.

\paragraph{Installation:} The latest version can be installed from CRAN like so:
 
\begin{knitrout}
\definecolor{shadecolor}{rgb}{0.969, 0.969, 0.969}\color{fgcolor}\begin{kframe}
\begin{alltt}
\hlcom{## Install}
\hlkwd{install.packages}\hlstd{(}\hlstr{"barplot3d"}\hlstd{)}
\hlcom{## Load}
\hlkwd{library}\hlstd{(barplot3d)}
\end{alltt}
\end{kframe}
\end{knitrout}

\noindent The latest development version can be cloned from GitHub but must be built from source:\\

\noindent \url{https://github.com/cpwardell/barplot3d}

\paragraph{Dependencies:} barplot3d depends on rgl\cite{rgl}.

\pagebreak
\tableofcontents
\pagebreak

\section{Overview}
\subsection{Introduction}
Insert introduction here


\section{Worked examples}
A number of examples are discussed in order of increasing complexity.
\subsection{Example 1 - description}

Note: fills from left to right, front to back 

You can manually edit the size/position of the window using commands like this:
par3d(windowRect=c(2004,866,2260,1122))

Explain how to save images:
creation of png output
Save your images to files if you wish
rgl.snapshot(filename="example.png")

Description in here
\subsection{Example 2 - description}
Description in here
\subsection{Example 3 - description}
Description in here

theta=50,phi=40 for Broad-style legoplots

\subsection{Example 4 - description}
Description in here









\section{Session Info}
\begin{knitrout}
\definecolor{shadecolor}{rgb}{0.969, 0.969, 0.969}\color{fgcolor}\begin{kframe}
\begin{alltt}
\hlkwd{sessionInfo}\hlstd{()}
\end{alltt}
\begin{verbatim}
## R version 3.2.2 (2015-08-14)
## Platform: x86_64-w64-mingw32/x64 (64-bit)
## Running under: Windows 7 x64 (build 7601) Service Pack 1
## 
## locale:
## [1] LC_COLLATE=English_United States.1252 
## [2] LC_CTYPE=English_United States.1252   
## [3] LC_MONETARY=English_United States.1252
## [4] LC_NUMERIC=C                          
## [5] LC_TIME=English_United States.1252    
## 
## attached base packages:
## [1] stats     graphics  grDevices utils     datasets  methods   base     
## 
## other attached packages:
## [1] knitr_1.11
## 
## loaded via a namespace (and not attached):
## [1] formatR_1.2    tools_3.2.2    highr_0.5      stringr_0.6.2 
## [5] evaluate_0.7.2
\end{verbatim}
\end{kframe}
\end{knitrout}

\printbibliography[heading=bibintoc] 



\end{document}
